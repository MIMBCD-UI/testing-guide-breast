%%%%%%%%%%%%%%%%%%%%%%%%%%%%%%%%%%%%%%%%%%%%%%%%%%%
%                                                 %
%                     SECTION                     %
%                                                 %
%%%%%%%%%%%%%%%%%%%%%%%%%%%%%%%%%%%%%%%%%%%%%%%%%%%

\section{Introduction}

This document aims to describe the protocol performing a set of tests in the scope of \hyperlink{https://github.com/MIMBCD-UI/prototype-breast-screening/releases/tag/v1.2.0-beta}{v1.2.0-beta} version from the \hyperlink{https://github.com/MIMBCD-UI/prototype-breast-cancer}{prototype-breast-cancer} repository of the \hyperlink{https://mimbcd-ui.github.io/}{MIMBCD-UI} project using traditional devices (mouse and keyboard). The goal of the test is to understand the user, performance, efficiency and efficacy metrics. With the session, the sessions will be recorded via video on the computer and using a record, heat-map and triggered event tools. It is guaranteed the confidentiality of the recordings, which will be used only for academic purposes.

Dividing the activity session into four distinct phases per each two activities representing two different scenarios (Single-Modality vs Multi-Modality). Each scenario will have three patients. In both two scenarios, by supporting our traditional devices, the interaction is made with mouse and keyboard. The first phase, is the \hyperlink{https://docs.google.com/forms/d/1cGmaCGZjeLJhUl_My2wxJ7gcpm7vQRxYhds6Ys0NoSc/edit?usp=sharing}{demographic questionnaire}, where we characterise the Radiologist profile. The second phase is the act of classifying those patients. Radiologists will classify each patient by using the \hyperlink{https://en.wikipedia.org/wiki/BI-RADS}{BIRADS} \cite{balleyguier2007birads}. On the third phase, we will do a small questionnaire at the end of each scenario using \hyperlink{https://en.wikipedia.org/wiki/NASA-TLX}{NASA-TLX}~\cite{ramkumar2017using}. Finally, the forth phase we will have a final survey regarding the Usability of each scenario. The well-known scale called \hyperlink{https, therefore, support this phase://en.wikipedia.org/wiki/System_usability_scale}{System Usability Scale (SUS)}~\cite{orfanou2015perceived}. For the user tests we used a two distinct prototype repositories \hyperlink{https://github.com/MIMBCD-UI/prototype-single-modality}{prototype-single-modality} and \hyperlink{https://github.com/MIMBCD-UI/prototype-multi-modality}{prototype-multi-modality}, both are "almost" mirrors of the \hyperlink{https://github.com/MIMBCD-UI/prototype-breast-cancer}{prototype-breast-cancer} with minor changes.