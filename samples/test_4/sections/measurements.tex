\section{Measurements}

Our measurements refers to user performance measured against specific performance goals necessary to satisfy requirements. \textit{Task} completion success rates, adherence to dialog scripts, error rates and subjective evaluations will be used. \textit{Time-to-completion} of \textit{tasks} will also be collected. The measures are as follows.

\hfill

%%%%%%%%%%%%%%%%%%%%%%%%%%%%%%%%%%%%%%%%%%%%%%%%%%%

The tests are intended to achieve the following measures:

%%%%%%%%%%%%%%%%%%%%%%%%%%%%%%%%%%%%%%%%%%%%%%%%%%%

\hfill

\begin{itemize}
\item BIRADS Classification;
\item Time measurement;
\item Number of clicks;
\item Number of errors;
\item Efficiency;
\item Difficulty;
\item Experience;
\end{itemize}

\hfill

%%%%%%%%%%%%%%%%%%%%%%%%%%%%%%%%%%%%%%%%%%%%%%%%%%%

To prioritise recommendations, a method for problem difficulty and degree severity classification will be used in the analysis of the collected data during evaluation process. The approach treats problem severity has a combination of several factors. Those factors are measuring the impact of the problem and the frequency of users experiencing issues during the evaluation.

\hfill

%%%%%%%%%%%%%%%%%%%%%%%%%%%%%%%%%%%%%%%%%%%%%%%%%%%

Through the questionnaire after the test session, we intend to obtain the answers to the following questions for each \textit{task}:

%%%%%%%%%%%%%%%%%%%%%%%%%%%%%%%%%%%%%%%%%%%%%%%%%%%

\hfill

\begin{itemize}
\item Difficulty of lesion classification;
\item Difficulty of interaction;
\item Difficulty translating;
\item Difficulty performing the several features;
\item Degree of classification;
\item Degree of interaction;
\end{itemize}

\hfill

%%%%%%%%%%%%%%%%%%%%%%%%%%%%%%%%%%%%%%%%%%%%%%%%%%%