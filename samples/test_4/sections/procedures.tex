\section{Procedures}

Participants will take part in the tests at our formed institution protocols (e.g. \hyperlink{http://hff.min-saude.pt/}{Hospital Fernando Fonseca (HFF)}) with the \hyperlink{https://github.com/MIMBCD-UI/prototype-breast-screening/releases/tag/v1.2.0-beta}{v1.2.0-beta} version of our \hyperlink{https://github.com/MIMBCD-UI/prototype-breast-screening/}{prototype-breast-screening}. The interaction with the system will be used in a typical \textbf{RR} environment. Note takers and data logger(s) will monitor the sessions for observation in the \textbf{RR}, connected by screen recording feed. The test sessions will be recorded and further analysed.

%%%%%%%%%%%%%%%%%%%%%%%%%%%%%%%%%%%%%%%%%%%%%%%%%%%
%                                                 %
%                     SECTION                     %
%                                                 %
%%%%%%%%%%%%%%%%%%%%%%%%%%%%%%%%%%%%%%%%%%%%%%%%%%%

\subsection{Briefing}

A presentation of the systems and it's use and capabilities will be made. Participants will be presented to the available interactions and will be explained how to interact with the prototype, underlining the limitations. The facilitator will brief the participants on the prototype application and instruct the participant that they are evaluating the application, rather than the facilitator evaluating the participant. Participants will sign an informed consent that acknowledges: the participation is voluntary, that participation can cease at any time, and that the session will be videotaped but their privacy of identification will be granted. The facilitator will ask the participant if they have any question.

%%%%%%%%%%%%%%%%%%%%%%%%%%%%%%%%%%%%%%%%%%%%%%%%%%%

%%%%%%%%%%%%%%%%%%%%%%%%%%%%%%%%%%%%%%%%%%%%%%%%%%%
%                                                 %
%                     SECTION                     %
%                                                 %
%%%%%%%%%%%%%%%%%%%%%%%%%%%%%%%%%%%%%%%%%%%%%%%%%%%

\subsection{Post-Task Questionnaire}

Our metrics will refer the user performance measured against specific performance goals necessary to satisfy several requirements of our system. For our \textbf{Post-Task Questionnaire} we will use \textbf{SUS} to measure the usability of our system each time a \textit{task} is completed. From a set of tasks (see \textbf{Tasks} section) we aim to cover several scenarios. Therefore, the \textbf{SUS} will allow the facilitator to quickly and easily assess the usability of a given scenario. This scale has several attributes \cite{bangor2008empirical} that make it a good choice for our clinical usability participants. Those attributes are as follows.

\hfill

%%%%%%%%%%%%%%%%%%%%%%%%%%%%%%%%%%%%%%%%%%%%%%%%%%%

List of the scale attributes:

%%%%%%%%%%%%%%%%%%%%%%%%%%%%%%%%%%%%%%%%%%%%%%%%%%%

\hfill

\begin{itemize}
  \item The survey is technology agnostic, making it flexible enough;
  \item The survey is relatively quick and easy to use;
  \item The survey provides a single score on a scale that is easily understood;
  \item The survey is nonproprietary, making it a cost effective tool;
\end{itemize}

\hfill

%%%%%%%%%%%%%%%%%%%%%%%%%%%%%%%%%%%%%%%%%%%%%%%%%%%

The facilitator will explain that the amount of time taken to complete the \textit{tasks} will be measured and that exploratory behaviour outside the \textit{task} flow should not occur until after task completion. At the beginning of each task, the participant will listen the task description from the facilitator and begin the task. \textit{Time-on-task} measurements begins when the participant starts the \textit{task}, measured until the end of each \textit{task}.

%%%%%%%%%%%%%%%%%%%%%%%%%%%%%%%%%%%%%%%%%%%%%%%%%%%

%%%%%%%%%%%%%%%%%%%%%%%%%%%%%%%%%%%%%%%%%%%%%%%%%%%
%                                                 %
%                     SECTION                     %
%                                                 %
%%%%%%%%%%%%%%%%%%%%%%%%%%%%%%%%%%%%%%%%%%%%%%%%%%%

\subsection{Training Session}

The participant will receive and overview the usability test procedure. However, the user will not receive information how to annotate and interact in all degrees of freedom. With the aim of disabling users to get their work done before the test tasks. It will take advantage of a "surprise" acknowledgement.

%%%%%%%%%%%%%%%%%%%%%%%%%%%%%%%%%%%%%%%%%%%%%%%%%%%
%                                                 %
%                     SECTION                     %
%                                                 %
%%%%%%%%%%%%%%%%%%%%%%%%%%%%%%%%%%%%%%%%%%%%%%%%%%%

\subsection{Execution of Tasks}

The \textit{tasks} were derived from test scenarios developed from \textbf{Case Studies}. Due to the range and extent of functionality provided by our prototype, and the short time from which each participant will be available, the \textit{tasks} are the most common and relatively complex of available functions. The \textit{tasks} are the identical for all participants of a given user role in the study.

The \textit{tasks} will be performed by several classes of radiology experience. Professionals from Radiology Seniors, Juniors and Interns will be performing these \textit{tasks}. On the \textbf{RR} the Radiologist is characterised~\cite{ehrlich2016patient, miglioretti2007radiologist} as a physician who examines and interpret Medical Imaging (MI) \cite{kobashi2017evaluation}, such as X-Rays, CT Scans or MRIs.

%%%%%%%%%%%%%%%%%%%%%%%%%%%%%%%%%%%%%%%%%%%%%%%%%%%

%%%%%%%%%%%%%%%%%%%%%%%%%%%%%%%%%%%%%%%%%%%%%%%%%%%
%                                                 %
%                     SECTION                     %
%                                                 %
%%%%%%%%%%%%%%%%%%%%%%%%%%%%%%%%%%%%%%%%%%%%%%%%%%%

\subsection{Post-Activity Questionnaire}

After completing all \textit{tasks} and scenarios, participants will be asked to complete a questionnaire to classify the prototype according to various parameters regarding the workload. To measure this, we will use the well known scale called \hyperlink{https://en.wikipedia.org/wiki/NASA-TLX}{NASA-TLX}~\cite{ramkumar2017using}. It consists of a set of six rating scales to evaluate the workload of the participant in a \textit{task} or a set of \textit{tasks}. \hyperlink{https://en.wikipedia.org/wiki/NASA-TLX}{NASA-TLX} is used in \textbf{Human-Computer Interaction (HCI)} research to identify users' performance, mental demand, emotion, etc. We will use this scale questionnaire to identify participants' workload during the various stages of the workflow.

%%%%%%%%%%%%%%%%%%%%%%%%%%%%%%%%%%%%%%%%%%%%%%%%%%%