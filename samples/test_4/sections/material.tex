%%%%%%%%%%%%%%%%%%%%%%%%%%%%%%%%%%%%%%%%%%%%%%%%%%%
%                                                 %
%                     SECTION                     %
%                                                 %
%%%%%%%%%%%%%%%%%%%%%%%%%%%%%%%%%%%%%%%%%%%%%%%%%%%

\section{Material}

For the material and apparatus, it is essential to capture the session apprehending the user interactions. In our case, we will record this interaction by using the \hyperlink{https://support.apple.com/quicktime}{QuickTime Player Version 10.4 (928.5.1)} to obtain all interactions. We will pair this video tool with a user watch tool called \hyperlink{https://www.hotjar.com/}{Hotjar}. This tool serves the purpose of using several logs of the interaction and gives us visualisation over it. Both instruments will help us to capture where are users interacting. By looking at the test participant's reactions, we find a lot of information regarding the prototype design.

The tools that we choose for the material and apparatus of this User Testing Guide are low-cost and easy to use. Our equipment is a cost-effective and, by using our laboratory materials, bringing it to the radiology room, we enable to capture not only what the user is doing on the screen, but on the body language supported by the interviews and observation.

\hfill

The material used in the test sessions for the user interface consists of:

\hfill

\begin{itemize}
  \item MacBook Pro: it will allow the user to interact with the keyboard and a wireless mouse;
  \item Wireless Mouse: it will allow the user to interact with a mouse and will complement the keyboard;
\end{itemize}

\hfill

\clearpage

%%%%%%%%%%%%%%%%%%%%%%%%%%%%%%%%%%%%%%%%%%%%%%%%%%%
%                                                 %
%                     SECTION                     %
%                                                 %
%%%%%%%%%%%%%%%%%%%%%%%%%%%%%%%%%%%%%%%%%%%%%%%%%%%

\subsection{Technical Details}

To produce this traditional environment, and since we can simulate with a laptop, the mouse and keyboard interaction, we are using a Microsoft Mobile Mouse 4000 together with the \hyperlink{https://www.apple.com/shop/buy-mac/macbook-pro}{MacBook Pro} (Retina, 13-inch, Early 2015) with a standard integrated keyboard on the laptop.

%%%%%%%%%%%%%%%%%%%%%%%%%%%%%%%%%%%%%%%%%%%%%%%%%%%
%                                                 %
%                     SECTION                     %
%                                                 %
%%%%%%%%%%%%%%%%%%%%%%%%%%%%%%%%%%%%%%%%%%%%%%%%%%%

\subsection{Software}

To track our user interactions across our system, we are using \hyperlink{https://www.hotjar.com/}{Hotjar}. This tool is an analytic package allowing us to follow our users remotely. It also provides two critical pieces of functionality, among others, that can aid in remote user testing. First of all, the heatmaps allow us to see where users are clicking, tapping and scrolling on our system. Second, it records a video playback of the entire user session. The tool shows evidence of being useful for our studies while we successfully used it in the past.

To record the task activities and the interview, we used \hyperlink{https://support.apple.com/downloads/quicktime}{QuickTime}~\cite{rowell2006internet}. The \hyperlink{https://support.apple.com/downloads/quicktime}{QuickTime} (\hyperlink{https://www.apple.com/}{Apple Computer}) tool is available for \hyperlink{https://www.apple.com/shop/buy-mac/macbook-pro}{MacBook Pro} to movie, audio and screen recording. Despite of have an overall of features, we just used it for our user's screen recording. It provides this functionalities at minimum requirements and compatible to our apparatus.