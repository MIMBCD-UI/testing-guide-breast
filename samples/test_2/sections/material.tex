\section{Material}

For the material and apparatus, it is essential to capture the session apprehending the user interactions. In our case, we will record this interaction by using the \hyperlink{https://support.apple.com/quicktime}{QuickTime Player Version 10.4 (928.5.1)} to obtain all interactions. We will pair this video tool with a user watch tool called \hyperlink{https://www.hotjar.com/}{Hotjar}. This tool serves the purpose of using several logs of the interaction and gives us visualisation over it. Both instruments will help us to capture where are users interacting. By looking at the test participant's reactions, we find a lot of information regarding the prototype design.

The tools that we choose for the material and apparatus of this User Testing Guida are low-cost and easy to use. Our equipment is a cost-effective and, by using our laboratory materials, bringing it to the radiology room, we enable to capture not only what the user is doing on the screen, but on the body language supported by the interviews and observation.

\hfill

The material used in the test sessions for the user interface consists of:

\hfill

\begin{itemize}
  \item MacBook Pro: it will allow the user to interact with the keyboard and a wireless mouse;
  \item Wireless Mouse: it will allow the user to interact with a mouse and will complement the keyboard;
\end{itemize}

\hfill

\clearpage

\subsection{Technical Details}

To produce this traditional environment, and since we can simulate with a laptop, the mouse and keyboard interaction, we are using a Microsoft Mobile Mouse 4000 together with the MacBook Pro (Retina, 13-inch, Early 2015) with a standard integrated keyboard on the laptop.