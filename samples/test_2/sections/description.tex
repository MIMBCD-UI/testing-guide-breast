\section{Description}

blabalaba

\subsection{Traditional Devices}

Traditional interaction remains the most common way to interact with user interfaces in a clinical environment. Unfortunately, most of this interaction is made by low profile equipment that makes users produce more errors and take more time interacting with those user interfaces.

%\begin{figure}[h]
%\caption{Example of a Freehand ROI Annotation}
%\centering
%\includegraphics[width=\textwidth]{}
%\end{figure}

As we can see in Figure 1, it describes our User Interface (UI), where the patient ("Anonymized1" for example) breast are on a small left column. The options are in a short row near of the viewport and described below. We also have the tabs where the user can change the patient. The centre viewport shows the DICOM Image, and it can be configured to display a number up to four DICOM images. The viewport has some text information on it (yellow) with the details of the metadata.

Manual annotation is adopted by us thanks to the Cornerstone Freehand ROI Annotation feature. According to the Cornerstone Library, the user can create an annotation by setting up consecutive landmarks around a Region of Interest (ROI). The markers finish a lesion annotation when it interconnects the historical. Additional features available in our User Interface (UI) includes on-demand increment of the number of landmarks, and throw transformations of the shape of an annotation.

\subsection{Interactive Buttons}

%\begin{figure}[h]
%\caption{All interactive buttons.}
%\centering
%\includegraphics[width=\textwidth]{}
%\end{figure}

The systems have several buttons (Figure 2) that allows the user to interact or access to a set of user interface features. The buttons are (from left to right):

\begin{itemize}
  \item WW/WC
  \item Invert
  \item Zoom
  \item Pan
  \item Stack Scroll
  \item Length Measurement
  \item Window Controller
  \item Freehand
\end{itemize}

\subsection{Usability Evaluation Technique}

\begin{table}[h]
\centering
\label{table:key_questions}
\begin{tabular}{l|l}
Number & Issues of Content Key Questions                    \\ \hline
1      & How do you perceive this activity?                 \\
2      & Could it be done in a more intuitive way?          \\
3      & What are the consequences?                         \\
4      & Why did you do as you did with this activity?      \\
5      & Is this activity relevant for you?                 \\
6      & Could you suggest another way to do this activity?
\end{tabular}
\caption{My caption}
\end{table}

\clearpage