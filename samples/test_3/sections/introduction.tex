%%%%%%%%%%%%%%%%%%%%%%%%%%%%%%%%%%%%%%%%%%%%%%%%%%%
%                                                 %
%                     SECTION                     %
%                                                 %
%%%%%%%%%%%%%%%%%%%%%%%%%%%%%%%%%%%%%%%%%%%%%%%%%%%

\section{Introduction}

This document aims to describe the protocol performing a set of tests in the scope of \hyperlink{https://github.com/MIMBCD-UI/prototype-breast-screening/releases/tag/v1.0.6-alpha}{v1.0.6-alpha} version from the \hyperlink{https://github.com/MIMBCD-UI/prototype-breast-cancer}{prototype-breast-cancer} repository of the \hyperlink{https://mimbcd-ui.github.io/}{Medical Imaging Multimodality of Breast Cancer Diagnosis User Interface (MIMBCD-UI)} project using a traditional devices (mouse and keyboard). The goal of the test is to understand the user, performance, efficiency and efficacy metrics. With the session, the sessions will be recorded via video on the computer and using a record, heat-map and triggered event tools. It is guaranteed the confidentiality of the recordings, which will be used only for academic purposes.

Dividing the activity session into three distinct phases of the three tasks. In all three tasks, by supporting our traditional devices, the interaction with mouse and keyboard. The first phase is the act of doing those tasks. On the second phase, we will do a small questionnaire at the end of each task using \hyperlink{https://en.wikipedia.org/wiki/NASA-TLX}{NASA-TLX}~\cite{ramkumar2017using}. Finally, the third phase we will have a final survey regarding the Usability of the system. The well-known scale called \hyperlink{https, therefore, support this phase://en.wikipedia.org/wiki/System_usability_scale}{System Usability Scale (SUS)}~\cite{orfanou2015perceived}.