\section{Tasks}

During our usability tests, we need to ask participants to provide a subjective assessment of their experience using our system. There are several widely used questionnaires giving us different prons-and-cons. However, in most cases, a \hyperlink{https://www.nngroup.com/articles/keep-online-surveys-short/}{single question instrument} \cite{sauro201210} is the right method for a quantitative usability testing. By taking less time and effort to answer, participants are pursuing to this phase after task while it is minimally disruptive.

\clearpage

In our \textbf{User Testing Guide} a set of tasks is necessary and carefully crafted. Our usability studies involve asking participants to perform the following tasks. By looking at what our user need to do with our system, our tasks are realistic as possible. We are not describing the exact steps participants need to take. We achieve that by avoiding the precise language used as labels in our system. The tasks are emotionally neutrals. And we did several \hyperlink{https://www.nngroup.com/articles/pilot-testing/}{pilot tests} to prevent misleading situations saving us from wasting resources by accidentally use a lousy task or from getting bad data. The tasks are as follows.

\hfill

%%%%%%%%%%%%%%%%%%%%%%%%%%%%%%%%%%%%%%%%%%%%%%%%%%%

List of stand alone tasks:

%%%%%%%%%%%%%%%%%%%%%%%%%%%%%%%%%%%%%%%%%%%%%%%%%%%

\hfill

\begin{itemize}
\item[] \textbf{Task 1:} Annotate the \textbf{US} modality from the \textbf{20160229} date;
\item[] \textbf{Task 2:} Annotate the last but one medical image of the \textbf{Breast} study description;
\item[] \textbf{Task 3:} Annotate the second \textbf{MR} frame number \textbf{10} of the last patient;
\end{itemize}

\hfill

%%%%%%%%%%%%%%%%%%%%%%%%%%%%%%%%%%%%%%%%%%%%%%%%%%%