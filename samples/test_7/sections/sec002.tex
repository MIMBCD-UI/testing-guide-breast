%%%%%%%%%%%%%%%%%%%%%%%%%%%%%%%%%%%%%%%%%%%%%%%%%%%
%                                                 %
%                     SECTION                     %
%                                                 %
%%%%%%%%%%%%%%%%%%%%%%%%%%%%%%%%%%%%%%%%%%%%%%%%%%%

\section{Description}
\label{sec:sec002}

This document describes our test plan for conducting our user tests during the development of the \hyperlink{https://breastscreening.github.io/}{BreastScreening} project and systems. The goals of the user testing phases include establishing a baseline of participant performance, establishing and validating participant performance measures, and identifying potential design concerns to be addressed in order to improve the efficiency, productivity, and end-user satisfaction within the development and introduction of \textit{AI-Assistive} methods inside the Radiology Room (RR), between others, for Medical Imaging (MI), or more precisely, the breast cancer diagnosis.

%%%%%%%%%%%%%%%%%%%%%%%%%%%%%%%%%%%%%%%%%%%%%%%%%%%

\hfill

The user test objectives are:

\begin{enumerate}
\item To determine design inconsistencies and issues within the UI and content areas;
\begin{enumerate}
\item \textbf{Navigation Errors};
\item \textbf{Presentation Errors};
\item \textbf{Control Usage Problems};
\end{enumerate}
\item Exercise the prototype under controlled test conditions with representative users;
\item Establish baseline user performance and user-satisfaction levels of the user interface for future usability evaluations;
\end{enumerate}

%%%%%%%%%%%%%%%%%%%%%%%%%%%%%%%%%%%%%%%%%%%%%%%%%%%

Potential sources of error may include: (a) \textbf{Navigation Errors:} failure to locate functions, excessive keystrokes to complete a function, failure to follow recommended screen flow; (b) \textbf{Presentation Errors:} failure to locate and properly act upon desired information in screens, selection errors due to labeling ambiguities; and (c) \textbf{Control Usage Problems:} improper toolbar or entry field usage. Data will be used to access whether usability goals regarding an effective, efficient, and well-received user interface have been achieved.

To verify our work, we identified measurable and explicit targets. By having several goals, including that a value percentage of the users should be able to operate the tasks without the need of help. On the same rate value, the user should be able to start and complete the medical diagnosis tasks over the system with little errors or mitigating those errors. Measuring the expected number of errors with a relation between pilot \hyperlink{https://github.com/MIMBCD-UI/testing-guide-breast/tree/master/samples/test_4}{early} tests. On the laboratory pilot tests we aim to test our prototypes with researchers.

Researchers are in the context of the system, and know well the functionalities, so that, we need to expect a percentage value over their results compared to clinicians and not the same benefits. Last but not least, both users (researchers and clinicians) should be able to understand in a similar time amount the meaning of all visible controls. By the similar amount of time, it is expected to have a variance of the percentage value between researchers and clinicians, as well as values of early tests, of the same value percentage of the early goals described in this paragraph.

We tested each objective in early laboratory and field tests, so that we could take the appropriate corrective actions. Also, we expect to run early field tests with researchers and clinicians to highlight issues that we overlooked and ignored during the prototyping phase. To support interaction use by the participants, we will try to emphasize several key factors on our user tests. The tasks must be simple, low intrusive, support for natural interaction and the system must always give visibility and the task current-state.