%%%%%%%%%%%%%%%%%%%%%%%%%%%%%%%%%%%%%%%%%%%%%%%%%%%
%                                                 %
%                     SECTION                     %
%                                                 %
%%%%%%%%%%%%%%%%%%%%%%%%%%%%%%%%%%%%%%%%%%%%%%%%%%%

\section{Description}
\label{sec:sec002}

%%%%%%%%%%%%%%%%%%%%%%%%%%%%%%%%%%%%%%%%%%%%%%%%%%%

We took impressions from another domain study~\cite{https://doi.org/10.13140/RG.2.2.32577.92001/1, https://doi.org/10.13140/RG.2.2.31319.62885/1}, where we applied several Human-Computer Interaction (HCI) techniques to extract important user needs and translate those needs into system requirements~\cite{calisto2017mimbcdui, calisto2019itmedex, https://doi.org/10.13140/rg.2.2.29816.70409, calisto2015aqmgasa}.
In this study, we follow the same logic to assess the user needs concerning the adoption of AI in a clinical environment.
Additionally, our research work involves the development of two categories of diagnostic systems that are different in terms of requirements.
First of all, an annotating system~\cite{10.1145/3132272.3134111, https://doi.org/10.13140/rg.2.2.14792.55049, 10.1145/3399715.3399744} is providing clinicians the ability to label~\cite{10.13140/RG.2.2.26143.51365} medical images and, consequently, generate various {\it datasets}~\cite{https://doi.org/10.13140/rg.2.2.16086.88649} that will be consumed by the AI models.
Second, the AI models will be trained by consuming these {\it datasets} and provide recommendation to clinicians as a second reader or as an autonomous patient diagnostic~\cite{calisto2019midaaiarfuv, https://doi.org/10.13140/rg.2.2.25412.68486, 10.13140/RG.2.2.32854.40000}.

Our research work deals with several challenges~\cite{https://doi.org/10.13140/RG.2.2.25718.65606, 10.13140/RG.2.2.28345.52322, 10.13140/RG.2.2.22788.07048, LesionsTypes}, where the goal is to address and surpass these challenges through an HCI approach.
We follow an human-centered perspective to understand the user needs and improve our novel systems through interaction.
Nevertheless, it is important to provide contextualization~\cite{https://doi.org/10.13140/RG.2.2.14314.95685/2} of the work done until now and the future directions that will be introduced in this document.

Although the current work under this research has already studied the preliminary acceptance and trust of AI systems~\cite{CALISTO2021102607}, more work should be done.
To strengthen the research evidence for such claims, further work must measure a more detailed technology adoption in the context of medical imaging diagnosis.
In this work, a continuing quest to ensure clinicians' acceptance of AI is an ongoing clinical challenge~\cite{10.13140/RG.2.2.21110.34888, UTA10, UTA11, UTA8, UTA9}.
However, this challenge has occupied researchers to such an extent that AI adoption in the clinical domain is now considered an opportunity.

In this UTA, we aim to demographically assess the main characteristics and user profiles of the medical imaging community.
Additionally, we will address the community acceptance to the AI topic so that we can understand the potential adoption of AI in the clinical workflow.
As a demographic and domain study, this UTA is the 8th (\href{https://github.com/MIMBCD-UI/meta/wiki/User-Research#test-8-demographic--domain-study-}{UTA8}) reporting guide of our research work and is an iteration~\cite{10.13140/RG.2.2.14555.52009/2} from the previous (\href{https://github.com/MIMBCD-UI/meta/wiki/User-Research#test-7-multi-modality-vs-assistant-}{UTA7}) reporting guide~\cite{https://doi.org/10.13140/rg.2.2.16566.14403/1}.
The previous iteration (\href{https://github.com/MIMBCD-UI/meta/wiki/User-Research#test-7-multi-modality-vs-assistant-}{UTA7}), titled as ``Assistant Introduction: User Testing Guide For A Comparison Between Multi-Modality and AI-Assisted Systems'', guided us through the introduction of an AI-assisted system in the clinical workflow.
However, we did not properly study the demographic characteristics that influence the adoption of AI in medical practice.
Hence, we will study the existent technology acceptance theories, adapting the Unified Theory of Acceptance and Use of Technology (UTAUT) to develop a model to evaluate the acceptance of AI in the clinical workflow.