%%%%%%%%%%%%%%%%%%%%%%%%%%%%%%%%%%%%%%%%%%%%%%%%%%%
%                                                 %
%                     SECTION                     %
%                                                 %
%%%%%%%%%%%%%%%%%%%%%%%%%%%%%%%%%%%%%%%%%%%%%%%%%%%

\section{Challenges}
\label{sec:sec010}

In addition to the challenges already highlighted in the presented document, we must accomplish the participation issues.
The difference in knowledge and expertise levels between the participants will inhibit communication and participation of participants in different ways.
Moreover, the factor that posed challenges to participants are involving them to a nominal adoption of consequences in the perceptions and practice, related ethical and self conflicts in presence of results.
Challenges are presented through this document and for practitioners to improve both study and research.

As in any large-scale measurement and evaluation effort, designing and validating the measures will be one of the most important and difficult challenges to overcome.
This document should be a stimulus to re-examine how we approach existing challenges and study some aspects of human behavior, such as clinicians' relationship with AI assistance and its role during medical imaging diagnosis, for instance, in breast cancer disease.
Against the lack of AI acceptance, this document provided the first detailed research study on the adoption of AI assistants, designed to mitigate the medical error on world-wide clinical institutions.
While we expect that some of our findings will not generalize beyond diagnostic systems, others provide early insight into the increasingly important role of safety, security, privacy, and trust in AI adoption and usage.

%%%%%%%%%%%%%%%%%%%%%%%%%%%%%%%%%%%%%%%%%%%%%%%%%%%