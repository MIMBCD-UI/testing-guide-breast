%%%%%%%%%%%%%%%%%%%%%%%%%%%%%%%%%%%%%%%%%%%%%%%%%%%
%                                                 %
%                     SECTION                     %
%                                                 %
%%%%%%%%%%%%%%%%%%%%%%%%%%%%%%%%%%%%%%%%%%%%%%%%%%%

\section{Metrics}
\label{sec:sec008}

Herein, we outline the theoretical analysis, the methodology employed to create the metrics, and the benefits that can be obtained.
In order to define the metrics, we need to generate a consensus from many perspectives as to what was important to measure and how the measures should be calculated.
On an ongoing basis, it is envisioned that these metrics would evolve and become much more comprehensive and complex; however, it is critical that the early-stage metrics be meaningful and feasibly generated from data that were clear, concise, and accessible.
In order to begin this process of measurement, reporting, and analysis with as much consensus as possible, our research team convened to work on identifying the first metrics.
One interesting debate focuses around whether safety, security, privacy, and trust indicators (Section~\ref{sec:sec003}) should be just that, an indicator, or an all-inclusive calculation, similar to the technology acceptance and adoption items.

In this research, we utilized the unprecedented opportunity presented by the need of technology acceptance at an international scale to better understand the factors affecting the adoption and use of AI systems in clinical environments.
We are particularly interested in investigating the role of safety, security, privacy, and trust indicators in the adoption context of AI assistants that supported clinicians on medical imaging diagnosis.
We also want to understand the effect of moderator variables (gender, age, education, and clinical experience) to relate with the achieved metrics of technology acceptance.

%%%%%%%%%%%%%%%%%%%%%%%%%%%%%%%%%%%%%%%%%%%%%%%%%%%