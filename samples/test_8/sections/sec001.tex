%%%%%%%%%%%%%%%%%%%%%%%%%%%%%%%%%%%%%%%%%%%%%%%%%%%
%                                                 %
%                     SECTION                     %
%                                                 %
%%%%%%%%%%%%%%%%%%%%%%%%%%%%%%%%%%%%%%%%%%%%%%%%%%%

\section{Introduction}
\label{sec:sec001}

%%%%%%%%%%%%%%%%%%%%%%%%%%%%%%%%%%%%%%%%%%%%%%%%%%%

During the breast cancer screening, missing cancers may not be identified until they are more advanced and less agreeable to treatment~\cite{Houssami2017}.
Artificial Intelligence (AI) in the medical workflows may help with this challenge~\cite{McKinney2020}.
Studies have demonstrated the ability of AI to meet the human's performance on various clinical tasks~\cite{info:doi/10.2196/10010, Topol2019}.
As a lack of medical professionals threatens the adequacy and availability of clinical services worldwide~\cite{doi:10.1002/j.2051-3909.2012.tb00169.x, rimmer2017radiologist}, the scalability of AI could improve to higher care.

The role of Human-AI Interaction (HAII) in healthcare delivery the appropriate settings in which it can be applied, and its impact on the quality of care have yet to be evaluated~\cite{Tschandl2020}.
There have been several attempts at addressing the effects of HAII across multiple workflows and different levels of clinical expertise~\cite{doi:10.1148/radiol.2019182627, doi:10.1148/ryai.2020200057}.
However, the use case of breast cancer diagnosis to address the effects from varied representations of AI-based supported by intelligent agents is still scarce.
This explains why it is an open topic research, and the motivation behind the proposed research of this User Testing and Analysis (UTA) guide.