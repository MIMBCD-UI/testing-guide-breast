%%%%%%%%%%%%%%%%%%%%%%%%%%%%%%%%%%%%%%%%%%%%%%%%%%%
%                                                 %
%                     SECTION                     %
%                                                 %
%%%%%%%%%%%%%%%%%%%%%%%%%%%%%%%%%%%%%%%%%%%%%%%%%%%

\section{Methodology}
\label{sec:sec003}

A substantial level of activity has witnessed the use of a wide range of exploratory techniques.
In fact, these exploratory techniques are examining many different systems in countless different contexts, to the extent that even the most cursory examination will reveal a variety of user perspectives, contexts, analysis, theories, and research methods~\cite{williams2015unified}.
Such situation has in turn led to an element of confusion, as it is often current to be forced of picking specific characteristics across a wide variety of models and theories.
In response to this confusion, and in order to harmonize the literature associated with acceptance of new systems, Venkatesh et al.~\cite{venkatesh2016unified} developed a unified model~\textendash~created and studied by these authors as UTAUT~\textendash~that brings together alternative views on user and innovation acceptance.

As a reporting guide, the document proposes the application of a model based on the UTAUT.
In this work, we are using this model as the constructs to study the determinants for adoption of AI systems in medical imaging diagnosis.
The idea is to test the model via Confirmatory Factor Analysis (CFA) and Structural Equation Modeling (SEM) while using clinicians' responses (expected n $>$ 300 clinicians) to a formulated UTAUT questionnaire.

Future results will show how an increased understanding of a vital role of safety, security, privacy, and trust in usage intention of intelligent agents in these medical fields.
It is expected that such results will show to this research how improvements of the workflow performance for a clinical AI system is a strong predictor of adoption, while medical professional experience ({\it i.e.}, Interns, Juniors, Middles and Seniors) and medical specialities ({\it e.g.}, Radiologists, Surgeons, Dermatologists, Neurologists, etc) are essential moderators of behavioral intention.
The future empirical findings will provide valuable theoretical contributions to HCI and AI researchers concerning the design and implementation of intelligent agents by explaining the reasons behind adoption and usage of AI systems in the clinical workflow.

%%%%%%%%%%%%%%%%%%%%%%%%%%%%%%%%%%%%%%%%%%%%%%%%%%%